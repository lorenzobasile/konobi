\documentclass{beamer}
\usetheme{metropolis} 
\usepackage{caption}
\usepackage[export]{adjustbox}
\usepackage[utf8]{inputenc}


\title{\huge Konobi}
\date{}
\subtitle{Software Development Methods project}
\author{Lorenzo Basile, Irene Brugnara, Roberto Corti, Arianna Tasciotti}

\begin{document}
	
  \maketitle
  \section{Introduction}
  
  \begin{frame}{Our project}
    The goal of our project was to develop a command line version of \textbf{Konobi}, a board game for two players. The project also contains a client-server version of the game, which allows the two players to play remotely.
    \vspace{0.7cm}
    \pause
    \\What tools did we use?
    \begin{itemize}
	    \item Java 15
	    \item Gradle
	    \item TravisCI
	    \item Git \& GitHub
    \end{itemize}

  \end{frame}
  
  \begin{frame}{Konobi}
	    Konobi is a drawless game and it can be played either on a go board or a chess board.
	    \\Two players, black and white, take turns at placing stones of their color on the board, starting with black. The aim of the players is to build chains of connected stones of their color.
	    \vspace{0.5cm}
	    \pause
	    \\The game is won by the first player who connects the two opposite edges of the board.
	    \begin{itemize}
	    \item Black: top $\leftrightarrow$ bottom
	    \item White: left $\leftrightarrow$ right
	    \end{itemize}
   \end{frame}
      
   \begin{frame}{Connections}
     Two like-colored stones can be:
     \vspace{0.5cm}
    \begin{columns}
			\column{0.5\textwidth}
			\begin{figure}
				\includegraphics[scale=0.35]{images/strong.png}
				\caption*{Strongly connected}
			\end{figure}
					
			\column{0.5\textwidth}
			\begin{figure}
				\includegraphics[scale=0.35]{images/weak.png}
				\caption*{Weakly connected}
			\end{figure}
		
	\end{columns}
	\vspace{0.7cm}
	A chain is a set of connected stones

      \end{frame}
      
      \begin{frame}{Placement rules}
     Not all moves are allowed:
     \begin{itemize}
     \item \textbf{Weak connections} to a certain stone are illegal unless it is impossible to make a placement that is both strongly connected to that stone and not weakly connected to another
     \item \textbf{Crosscut} placements are always illegal
     \end{itemize}
    \begin{columns}
			\column{0.5\textwidth}
			\begin{figure}
				\includegraphics[scale=0.35]{images/legal_weak.png}
				\caption*{Legal weak connection}
			\end{figure}
					
			\column{0.5\textwidth}
			\begin{figure}
				\includegraphics[scale=0.35]{images/crosscut.png}
				\caption*{Crosscut placement}
			\end{figure}
		
	\end{columns}
      \end{frame}
      
      \begin{frame}{Additional rules}
     \begin{itemize}
     \item \textbf{Pie rule}: at his first move, white can decide to switch colors with black instead of making a move.
     \item \textbf{Mandatory pass}: if a player cannot make a move (because of placement restrictions), he has to pass. It is guaranteed that at least one player can make a move.
     \end{itemize}
      \end{frame}
      
      
\section{Basic entities}
\begin{frame}{Cell}
% Lorenzo parla dei pacchetti
% inizio a parlare delle classi color, position, cell, board che stanno nel pacchetto Entities

\begin{columns}
\column{0.5\textwidth}
\texttt{Cell} represents the basic building block of the board 
% A cell has three fields: the position of the cell in the board, a boolean isOccupied which tells whether the cell contains a stone or not, and a Color which is the color of the stone if present. Position is a class with two fields x,y; color is an enum with two values BLACK, WHITE (it's used to mark Players as well as stones, as we will see)

\column{0.5\textwidth}
\begin{figure}
	\includegraphics[scale=0.4]{images/cell-class.png}
\end{figure}

\end{columns}


\end{frame}



\begin{frame}{Cell}
	% the constructor of Cell creates an empty cell
	When a cell is constructed it is empty: no color is associated to it and \texttt{isOccupied} is \texttt{False}. When a stone is placed in the cell a color is set and \texttt{isOccupied} becomes \texttt{True}.
	\vspace{0.5cm}
	% The methods setColor() and reset() are used for checking the rules, as we will see later. For this reason, Color is not declared as final, even if in the game once a stone is placed it cannot be moved afterwards. The position is final because a cell is associated with a fixed position in the board.

Development history:
\begin{itemize}
\item From value \texttt{NONE} in enum \texttt{Color} to field \texttt{isOccupied} in class \texttt{Cell}
\item Removed \texttt{Stone} data class
\end{itemize}
% At the beginning, the Color enum had also a NONE value, but then we thought that a Player cannot have NONE as a value for his color so we moved NONE into a new field of Cell which is a boolean called isOccupied, and the color can be only WHITE or BLACK, which is also more logical.

% at the beginning we had a class Stone but we removed it because it was a data class (only had setters, ..) and it basically coincided with Color

\end{frame}


\begin{frame}{Board}

A \texttt{Board} is represented by a set of cells and extends \texttt{HashSet<Cell>} by overriding the \texttt{dimension()} method

\vspace{0.5cm}
Reasons for this choice of data structure:
\begin{itemize}
    \item Usage of streams
    \item \texttt{Position} as attribute of \texttt{Cell}    
\end{itemize}



% The board is represented by a set of cells. The other option we considered was a 2-dimensional array. In this way we can easily use streams to perform operations on this set of cells (we will see with connections). And in addition, the Position is a property (a field) of a Cell and not a pair of indexes of a matrix.

%The constructor of Board creates a set of empty cells

\end{frame}


\section{Connections}
% These were the entities that make our game. Then we implemented the relationships between these entities, which are the connections, i.e. strong and weak connections between stones.

\begin{frame}{Strong connections}
	\begin{itemize}
		\item We implemented a concept of orthogonal adjacency which only depends on the relative positions of two cells: their euclidean distance must be 1
		% it's a relationship between cells, not stones
		\item Given the set of stones that are orthogonally adjacent to a given stone, by filtering only those with the same color we obtain the set of strongly connected neighbors
	\end{itemize}

\begin{columns}
	\column{0.3\textwidth}
	in class \texttt{Position}:
	
	  \vspace{0.9cm}
	in class \texttt{Cell}:
	
	  \vspace{0.9cm}
	in class \texttt{Board}:

	\column{0.7\textwidth}

\begin{figure}
	\includegraphics[scale=0.2]{images/connections-position.png}
	
  \vspace{0.2cm}
	\includegraphics[scale=0.2]{images/connections-cell.png}

  \vspace{0.2cm}
	\includegraphics[scale=0.2]{images/connections-board.png}
\end{figure}
\end{columns}

	
\end{frame}

\begin{frame}{Weak connections}

	\begin{itemize}
		\item Same concept applies for weak connections: two cells are diagonally adjacent if their squared euclidean distance is 2	
		\item To obtain the set of weak neighbors, in this case it is also necessary to filter out diagonally adjacent stones with common strong neighbors
		% in this case it is not sufficient to filter based on color 
	\end{itemize}
	in class \texttt{Board}:
\begin{figure}
	\includegraphics[scale=0.25]{images/connections-common.png}
\end{figure}

\end{frame}



\begin{frame}{Tests}
	Different test cases for orthogonal and diagonal adjacency:
\begin{columns}
\column{0.33\textwidth}
	\begin{figure}
		\includegraphics[scale=0.15]{images/diagonal_inner.png}
		\\inner cell
	\end{figure}
\column{0.33\textwidth}
	\begin{figure}
		\includegraphics[scale=0.15]{images/orthogonal_edge.png}
		\\cell on edge
	\end{figure}
\column{0.33\textwidth}
	\begin{figure}
	\includegraphics[scale=0.15]{images/diagonal_corner.png}
    \\cell on corner
\end{figure}
\end{columns}
	\vspace{0.9cm}
	Example of tests for strong and weak connection: 
	\begin{columns}
		\column{0.5\textwidth}
		\begin{figure}
			\includegraphics[scale=0.15]{images/test-strong-connection.png}
			
			the two stones are \\\textit{not} strongly connected
		\end{figure}
		
		\column{0.5\textwidth}
		\begin{figure}
			\includegraphics[scale=0.15]{images/test-weak-connection.png}
			
			the two stones are \\\textit{not} weakly connected
		\end{figure}
	\end{columns}
\end{frame}



\section{Rules}

\begin{frame}{Placement rules: how to implement them?}
	Initially, a \texttt{Rules} class was implemented
	\vspace{0.2cm}
	%\\ \small From commit \href{https://github.com/lorenzobasile/konobi/blob/eaba694d6662a1b6803f7f22943636176281e572/src/main/java/konobi/Rules.java}{eaba694}
	\begin{figure}
		\includegraphics[scale=0.28]{images/rules-class.png}
	\end{figure}
\end{frame}

\begin{frame}{Rules package}
	
	Later we realized that there would be the possibility to abstract...
	\begin{figure}
		\includegraphics[scale=0.4]{images/rules-uml.jpg}
	\end{figure}
	
	For a given \texttt{Board}, the \texttt{isValid} method will check if it is legal to place a stone of a given \texttt{Color} in the \texttt{Cell}. \\
	\vspace{0.1cm}
	A \texttt{Rule} interface will allow the possibility to add new rules on the game.
	
\end{frame}

\begin{frame}[t]{How they were implemented }
	\begin{itemize}
		\item \texttt{WeakConnectionRule}
		\vspace{0.2cm}
		\
		\begin{figure}
			\includegraphics[scale=0.16, center]{images/weakconnectionrule-code.png}
		\end{figure}
		\item \texttt{CrossCutRule}
		\vspace{0.2cm}
		\begin{figure}
			\includegraphics[scale=0.2, center]{images/crosscutrule-code.png}
		\end{figure}
	\end{itemize}
\end{frame}

\begin{frame}[t]{How they were tested }
	
	We implemented \texttt{WeakConnectionRule} and \texttt{CrossCutRule} by following TDD principles based on some examples...\\
	\vspace{0.1cm}
	\textbf{Legal moves}
	\begin{columns}
		\column{0.5\textwidth}
		%codice
		\begin{figure}[t]
			\includegraphics[scale=0.2]{images/test-legal-weak.png}
		\end{figure}
		\column{0.3\textwidth}
		%esempio
		\begin{figure}[t]
			\includegraphics[scale=0.15]{images/legal-move.png}
		\end{figure}
		
	\end{columns}
	
	\textbf{Illegal moves}
	\begin{columns}
		\column{0.3\textwidth}
		%codice
		\begin{figure}[t]
			\includegraphics[scale=0.2]{images/illegal-move.png}
		\end{figure}
		\column{0.5\textwidth}
		%esempio
		\begin{figure}
			\includegraphics[scale=0.2]{images/test-illegal-weak.png}
		\end{figure}
		
	\end{columns}
	
	
	
	
\end{frame}

\begin{frame}{A new entity comes: Referee }
	\begin{columns}
		\column{0.5\textwidth}
		Given the logic of a valid move in \texttt{Rules} package, our aim was to group together all the methods needed to check if:
		\vspace{0.4cm}
		\begin{itemize}
			\item a given move is legal w.r.t. \texttt{WeakConnectionRule} and \texttt{CrossCutRule}
			\vspace{0.25cm}
			\item a winning chain is present
			\vspace{0.25cm}
			\item the current player has to pass
		\end{itemize}
		
		\column{0.5\textwidth}
		
		\includegraphics[scale=0.27]{images/referee-class.jpg}
		
	\end{columns}
\end{frame}

\begin{frame}{How to find a chain}
	\texttt{validateChain} method is used to find a chain of a given \texttt{Color} in the game's board.
	
	
	\begin{figure}
		\includegraphics[scale=0.2, width=7cm]{images/chainsearch-code.png}
	\end{figure}
	
	\textbf{Recursive approach}: starting from all stones already placed on the start edge, we explore the set of their connected stones inside the board until we find a stone in the end edge.
\end{frame}


\section{InputOutput}

\begin{frame}{InputHandler and Display}
	Classes \texttt{InputHandler} and \texttt{Display} take care of game I/O.

	\begin{itemize}
		\item \texttt{Display}: interaction with players
	 	\item \texttt{InputHandler}: management of player inputs
	\end{itemize}
	
	INSERIRE UML INPUTOUTPUT E EXCEPTIONS
\end{frame}


\section{Game dynamics}

\begin{frame}{Overview}
	  \begin{center}
     		\includegraphics[scale=0.25]{images/game_uml.png}
     	\end{center}

\end{frame}

\begin{frame}{GameState}
     
     \texttt{GameState} class keeps track of current state of the game.
     
     METTERE UML DEI METODI PUBBLICI PIUTTOSTO CHE QUESTO DEI MEMBRI.
     \begin{center}
     	\includegraphics[scale=0.32]{images/gamestate.png}
     \end{center}
    
     \vspace{0.4cm}
   \texttt{GameState} acts as an intermediary between \texttt{Referee} and \texttt{Board} on one side and the higher-level class \texttt{Game} on the other 	side.
     
\end{frame}

\begin{frame}{Game}	 
	Based on the directives of \texttt{GameState}, \texttt{Game} manages the interactions with players.
	\\
	\vspace{0.5cm}
     	\pause
	\includegraphics[scale=0.26]{images/play.png}
   	\pause
	\includegraphics[scale=0.26]{images/singleturn.png}
	\pause
	\includegraphics[scale=0.26]{images/regularmove.png}
\end{frame}

\begin{frame}{GameInitializer}

	\texttt{GameInitializer} abstracts the initialization of the \texttt{Game}.
	
	AGGIUNGERE UML METODI DI GAMEINITIALIZER

\end{frame}


\section{Running the game}

\begin{frame}{Versions of Konobi}
INSERT UML HERE
Konobi can be played by the two players on the same terminal or in a Client-Server version, with the two players connecting through \texttt{telnet} to a Server running the game.
\end{frame}

\begin{frame}{Comparison between Console and C/S}
\begin{columns}
\column{0.5\linewidth}
\includegraphics[scale=0.27]{images/mainCo.png}
\column{0.5\linewidth}
Console version: a \texttt{Game} is initialized and its \texttt{play} method is called.
\end{columns}
\vspace{0.5cm}
\begin{columns}
\column{0.5\linewidth}
\includegraphics[scale=0.21]{images/mainCS.png}
\column{0.5\linewidth}
Client-Server version: The server creates the socket and waits for two clients to connect to it. The port number can be decided using command-line arguments.
\end{columns}\end{frame}
\begin{frame}{Inheritance in GameInitializer}

\end{frame}









     
\end{document}
